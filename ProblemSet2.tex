%%%%%%%%%%%%%%%%%%%%%%%%%%%%%%%%%%%%%%%%%
% Lachaise Assignment
% LaTeX Template
% Version 1.0 (26/6/2018)
%
% This template originates from:
% http://www.LaTeXTemplates.com
%
% Authors:\textsl{}
% Marion Lachaise & François Févotte
% Vel (vel@LaTeXTemplates.com)
%
% License:
% CC BY-NC-SA 3.0 (http://creativecommons.org/licenses/by-nc-sa/3.0/)
% 
%%%%%%%%%%%%%%%%%%%%%%%%%%%%%%%%%%%%%%%%%

%----------------------------------------------------------------------------------------
%	PACKAGES AND OTHER DOCUMENT CONFIGURATIONS
%----------------------------------------------------------------------------------------

\documentclass{article}
%%%%%%%%%%%%%%%%%%%%%%%%%%%%%%%%%%%%%%%%%
% Lachaise Assignment
% Structure Specification File
% Version 1.0 (26/6/2018)
%
% This template originates from:
% http://www.LaTeXTemplates.com
%
% Authors:
% Marion Lachaise & François Févotte
% Vel (vel@LaTeXTemplates.com)
%
% License:
% CC BY-NC-SA 3.0 (http://creativecommons.org/licenses/by-nc-sa/3.0/)
% 
%%%%%%%%%%%%%%%%%%%%%%%%%%%%%%%%%%%%%%%%%

%----------------------------------------------------------------------------------------
%	PACKAGES AND OTHER DOCUMENT CONFIGURATIONS
%----------------------------------------------------------------------------------------

\usepackage{amsmath,amsfonts,stmaryrd,amssymb} % Math packages
\usepackage{enumerate} % Custom item numbers for enumerations

\usepackage[ruled]{algorithm2e} % Algorithms

\usepackage[framemethod=tikz]{mdframed} % Allows defining custom boxed/framed environments

\usepackage{listings} % File listings, with syntax highlighting
\lstset{
	basicstyle=\ttfamily, % Typeset listings in monospace font
}

%----------------------------------------------------------------------------------------
%	DOCUMENT MARGINS
%----------------------------------------------------------------------------------------

\usepackage{geometry} % Required for adjusting page dimensions and margins

\geometry{
	paper=a4paper, % Paper size, change to letterpaper for US letter size
	top=2.5cm, % Top margin
	bottom=3cm, % Bottom margin
	left=2.5cm, % Left margin
	right=2.5cm, % Right margin
	headheight=14pt, % Header height
	footskip=1.5cm, % Space from the bottom margin to the baseline of the footer
	headsep=1.2cm, % Space from the top margin to the baseline of the header
	%showframe, % Uncomment to show how the type block is set on the page
}

%----------------------------------------------------------------------------------------
%	FONTS
%----------------------------------------------------------------------------------------

\usepackage[utf8]{inputenc} % Required for inputting international characters
\usepackage[T1]{fontenc} % Output font encoding for international characters

\usepackage{XCharter} % Use the XCharter fonts

%----------------------------------------------------------------------------------------
%	COMMAND LINE ENVIRONMENT
%----------------------------------------------------------------------------------------

% Usage:
% \begin{commandline}
%	\begin{verbatim}
%		$ ls
%		
%		Applications	Desktop	...
%	\end{verbatim}
% \end{commandline}

\mdfdefinestyle{commandline}{
	leftmargin=10pt,
	rightmargin=10pt,
	innerleftmargin=15pt,
	middlelinecolor=black!50!white,
	middlelinewidth=2pt,
	frametitlerule=false,
	backgroundcolor=black!5!white,
	frametitle={Command Line},
	frametitlefont={\normalfont\sffamily\color{white}\hspace{-1em}},
	frametitlebackgroundcolor=black!50!white,
	nobreak,
}

% Define a custom environment for command-line snapshots
\newenvironment{commandline}{
	\medskip
	\begin{mdframed}[style=commandline]
}{
	\end{mdframed}
	\medskip
}

%----------------------------------------------------------------------------------------
%	FILE CONTENTS ENVIRONMENT
%----------------------------------------------------------------------------------------

% Usage:
% \begin{file}[optional filename, defaults to "File"]
%	File contents, for example, with a listings environment
% \end{file}

\mdfdefinestyle{file}{
	innertopmargin=1.6\baselineskip,
	innerbottommargin=0.8\baselineskip,
	topline=false, bottomline=false,
	leftline=false, rightline=false,
	leftmargin=2cm,
	rightmargin=2cm,
	singleextra={%
		\draw[fill=black!10!white](P)++(0,-1.2em)rectangle(P-|O);
		\node[anchor=north west]
		at(P-|O){\ttfamily\mdfilename};
		%
		\def\l{3em}
		\draw(O-|P)++(-\l,0)--++(\l,\l)--(P)--(P-|O)--(O)--cycle;
		\draw(O-|P)++(-\l,0)--++(0,\l)--++(\l,0);
	},
	nobreak,
}

% Define a custom environment for file contents
\newenvironment{file}[1][File]{ % Set the default filename to "File"
	\medskip
	\newcommand{\mdfilename}{#1}
	\begin{mdframed}[style=file]
}{
	\end{mdframed}
	\medskip
}

%----------------------------------------------------------------------------------------
%	NUMBERED QUESTIONS ENVIRONMENT
%----------------------------------------------------------------------------------------

% Usage:
% \begin{question}[optional title]
%	Question contents
% \end{question}

\mdfdefinestyle{question}{
	innertopmargin=1.2\baselineskip,
	innerbottommargin=0.8\baselineskip,
	roundcorner=5pt,
	nobreak,
	singleextra={%
		\draw(P-|O)node[xshift=1em,anchor=west,fill=white,draw,rounded corners=5pt]{%
		Question \theQuestion\questionTitle};
	},
}

\newcounter{Question} % Stores the current question number that gets iterated with each new question

% Define a custom environment for numbered questions
\newenvironment{question}[1][\unskip]{
	\bigskip
	\stepcounter{Question}
	\newcommand{\questionTitle}{~#1}
	\begin{mdframed}[style=question]
}{
	\end{mdframed}
	\medskip
}

%----------------------------------------------------------------------------------------
%	WARNING TEXT ENVIRONMENT
%----------------------------------------------------------------------------------------

% Usage:
% \begin{warn}[optional title, defaults to "Warning:"]
%	Contents
% \end{warn}

\mdfdefinestyle{warning}{
	topline=false, bottomline=false,
	leftline=false, rightline=false,
	nobreak,
	singleextra={%
		\draw(P-|O)++(-0.5em,0)node(tmp1){};
		\draw(P-|O)++(0.5em,0)node(tmp2){};
		\fill[black,rotate around={45:(P-|O)}](tmp1)rectangle(tmp2);
		\node at(P-|O){\color{white}\scriptsize\bf !};
		\draw[very thick](P-|O)++(0,-1em)--(O);%--(O-|P);
	}
}

% Define a custom environment for warning text
\newenvironment{warn}[1][Warning:]{ % Set the default warning to "Warning:"
	\medskip
	\begin{mdframed}[style=warning]
		\noindent{\textbf{#1}}
}{
	\end{mdframed}
}

%----------------------------------------------------------------------------------------
%	INFORMATION ENVIRONMENT
%----------------------------------------------------------------------------------------

% Usage:
% \begin{info}[optional title, defaults to "Info:"]
% 	contents
% 	\end{info}

\mdfdefinestyle{info}{%
	topline=false, bottomline=false,
	leftline=false, rightline=false,
	nobreak,
	singleextra={%
		\fill[black](P-|O)circle[radius=0.4em];
		\node at(P-|O){\color{white}\scriptsize\bf i};
		\draw[very thick](P-|O)++(0,-0.8em)--(O);%--(O-|P);
	}
}

% Define a custom environment for information
\newenvironment{info}[1][Info:]{ % Set the default title to "Info:"
	\medskip
	\begin{mdframed}[style=info]
		\noindent{\textbf{#1}}
}{
	\end{mdframed}
}
 % Include the file specifying the document structure and custom commands

%----------------------------------------------------------------------------------------
%	ASSIGNMENT INFORMATION
%----------------------------------------------------------------------------------------

\title{MKT 7317 Problem Set 2} % Title of the assignment

\author{Jin Miao\\ \texttt{jxm190071@utdallas.edu}} % Author name and email address

\date{University of Texas at Dallas \\ \today} % University, school and/or department name(s) and a date

%----------------------------------------------------------------------------------------
\usepackage{setspace}
\usepackage{mathtools}  
\usepackage{booktabs}
\usepackage{graphicx} 
\usepackage{caption,subcaption}
%\usepackage[showframe]{geometry}
\usepackage{listings}
\usepackage{bbold}
\usepackage{float}
\usepackage{amsmath,amssymb}
\newcommand{\indep}{\perp \!\!\! \perp}

\usepackage[utf8]{inputenc}
\usepackage[english]{babel}
\setlength{\parindent}{2em}
\setlength{\parskip}{1em}
\renewcommand{\baselinestretch}{1}

\usepackage{amsmath, amsthm, amssymb, mathtools, thmtools}
\usepackage{hyperref, cleveref}

\begin{document}
\doublespacing
\maketitle % Print the title

%----------------------------------------------------------------------------------------
%	INTRODUCTION
%----------------------------------------------------------------------------------------

\section{Nested Logit with Three Alternatives}

\subsection*{Question (a)}
Given $\rho = 1$, we have 
\begin{equation*}
F(\epsilon_1, \epsilon_2) = e^{-e^{-\epsilon_1} - e^{-\epsilon_2}}
\end{equation*}
\begin{equation*}
\begin{cases}
F_1(\epsilon_1) & = \lim_{\epsilon_2 \rightarrow \infty} F(\epsilon_1, \epsilon_2) = \lim_{\epsilon_2 \rightarrow \infty} e^{-e^{-\epsilon_1} - e^{-\epsilon_2}}
= e^{-e^{-\epsilon_1}} \\
F_2(\epsilon_2) & = \lim_{\epsilon_1 \rightarrow \infty} F(\epsilon_1, \epsilon_2) = \lim_{\epsilon_1 \rightarrow \infty} e^{-e^{-\epsilon_1} - e^{-\epsilon_2}}
= e^{-e^{-\epsilon_2}} \\
\end{cases}
\end{equation*}

Given $F{\epsilon_1, \epsilon_2} = F_1(\epsilon_1) F_2(\epsilon_2)$, we conclude that $\epsilon_1 \indep \epsilon_2 $. 

We can also derive the probability distribution function as follows. 

\begin{equation*}
f(\epsilon_1, \epsilon_2) = \frac{\partial^2 F(\epsilon_1, \epsilon_2)}{\partial \epsilon_1 \partial \epsilon_2} = \epsilon_1 \epsilon_2 e^{-\epsilon_1} e^{\epsilon_2} e^{-e^{-\epsilon_1} - e^{-\epsilon_2}}
\end{equation*}
\begin{equation*}
f_1(\epsilon_1) = \frac{d F_1(\epsilon_1)}{d \epsilon_1 } = \epsilon_1 e^{-\epsilon_1} e^{-e^{-\epsilon_1}}
\end{equation*}
\begin{equation*}
f_2(\epsilon_2) = \frac{d F_2(\epsilon_2)}{d \epsilon_2 } = \epsilon_2 e^{-\epsilon_2} e^{-e^{-\epsilon_2}}
\end{equation*}

Given $f(\epsilon_1, \epsilon_2) = f_1(\epsilon_1) f_2(\epsilon_2)$, we conclude that $\epsilon_1 \indep \epsilon_2 $. 

\subsection*{Question (b)}
\addtolength{\jot}{2pt}
\begin{equation*}
\begin{split}
Pr\bigg(\text{i chooses 0}\bigg) & = Pr\bigg(U_{i0} \geq U_{i1} \text{ and } U_{i0} \geq U_{i2}\bigg) \\
& = Pr\bigg(\delta_0 + \epsilon_0 \geq \delta_1 + \epsilon_{i1} \text{ and } \delta_0 + \epsilon_0 \geq \delta_2 + \epsilon_{i2}\bigg) \\
& = Pr\bigg(\epsilon_{i1} \leq \delta_0 - \delta_1  + \epsilon_0 \text{ and } \epsilon_{i2} \leq \delta_0 - \delta_2 + \epsilon_0 \bigg)\\ 
& = \int F\bigg(\delta_0 - \delta_1  + \epsilon_0, \delta_0 - \delta_2  + \epsilon_0) dF(\epsilon_{0}\bigg) \\
& = \int_{-\infty}^{\infty} exp \bigg \{ - \bigg [exp\bigg(- \frac{ \delta_0 - \delta_1  + \epsilon_0 }{\rho} \bigg) + exp \bigg(- \frac{ \delta_0 - \delta_2 + \epsilon_0 }{\rho} \bigg) \bigg]^{\rho} \bigg \} d exp[-exp(\epsilon_0)] \\
& = \int_{-\infty}^{\infty} exp\bigg\{  -exp(\epsilon_0) \bigg[ exp \big( - \frac{ \delta_0 - \delta_1  }{ \rho } \big) + exp \big( - \frac{ \delta_0 - \delta_2 }{ \rho } \big) \bigg]^{\rho} \bigg\} dexp[-exp(\epsilon_0)]  \\
& = \frac{ 1 }{ \bigg[ exp \big( - \frac{ \delta_0 - \delta_1  }{ \rho } \big) + exp \big( - \frac{ \delta_0 - \delta_2 }{ \rho } \big) \bigg]^{\rho} + 1 }  \\
& = \frac{ 1 }{ exp(-\delta_0) \bigg[ exp \big( \frac{ \delta_1  }{ \rho } \big) + exp \big( \frac{ \delta_2 }{ \rho } \big) \bigg]^{\rho} + 1 }  \\
& = \frac{ exp(\delta_0) }{ \bigg[ exp \big( \frac{ \delta_1  }{ \rho } \big) + exp \big( \frac{ \delta_2 }{ \rho } \big) \bigg]^{\rho} + exp(\delta_0) }  \\
\end{split}
\end{equation*}

\subsection*{Question (c)}

Given that "1" (red bus) and "2" (blue bus) are in the same nest, and that "0" (car) is in another nest, we can derive the within-nest choice probability as 
\begin{equation}
\begin{split}
& Pr\bigg(\text{i chooses 1 \bigg| i does not chooses 0}\bigg) \\
& = Pr\bigg( U_{i1} \geq U_{i2} \bigg) \\
& = Pr\bigg(\delta_1 + \epsilon_{i1} \geq \delta_0 + \epsilon_{i0} \bigg) \\
& = \int_{-\infty}^{\infty} F_1\bigg( \epsilon, \delta_1 - \delta_2 + \epsilon \bigg) d\epsilon
\end{split}
\end{equation}
where $F_1\bigg( \epsilon_1, \epsilon_2 \bigg) = \frac{\partial F(\epsilon_1, \epsilon_2)}{\partial \epsilon_1} = F(\epsilon_1, \epsilon_2) \bigg( exp\big( -\frac{\epsilon_1}{\rho} \big) + exp\big( -\frac{\epsilon_1}{\rho} \big) \bigg)^{\rho -1} exp\bigg( -\frac{\epsilon_1}{\rho} \bigg) $

Then denote $K = exp\big( -\frac{\delta_1 - \delta_2}{\rho} \big) + 1$ and $t = exp\big( -\frac{\epsilon}{\rho} \big)$, we can simplify the integral as follows
\begin{equation}
\begin{split}
& Pr\bigg(\text{i chooses 1 \bigg| i does not chooses 0}\bigg) \\
& = \frac{\rho}{K} \int_{t = 0}^{+\infty} exp\bigg( -(tK)^{\rho} \bigg) (tK)^{\rho - 1} d(tK) \\
& = -\frac{1}{K} \int_{t = 0}^{+\infty} exp\bigg( -(tK)^{\rho} \bigg)  d\bigg( -(tK)^{\rho} \bigg) \\
& = \frac{1}{K} \\
& = \frac{1}{1 + exp\big( -\frac{\delta_1 - \delta_2}{\rho} \big) } \\
& = \frac{ exp\big( \frac{\delta_1}{\rho} \big) }{  exp\big( \frac{\delta_1}{\rho} \big) + exp\big( \frac{\delta_2}{\rho} \big) }
\end{split}
\end{equation}



\section{Price Elasticities in Logit Demand Model}
\addtolength{\jot}{-5pt}
In the system of homogeneous consumers, the Marshallian demand is proportionate to the market share $s_j$, which is approximate to the individual choice probability $Pr(i \rightarrow j)$. Given the consideration set $\mathcal{J}_t$, the regularity condition holds that for any alternative $j \in \mathcal{J}_t$, the price after infinitesimal changes is still smaller or equal to the income, $p_j + \Delta p \leq m$. 

\begin{equation*}
\frac{\partial Pr(i \rightarrow j)}{\partial p_c} \frac{p_c}{s_j} = 
\begin{cases}
- \alpha p_j (1 - s_j) & \text{if    } j = c \\
\alpha p_c s_c & \text{if    } j \neq c \\
\end{cases}
\end{equation*}

This is the legitimate Marshallian own and cross price elasticities by using choice probability as the infinitesimal changes in prices $p_c$ leads to changes in the individual choice probabilities $Pr(i \rightarrow j)$, which then translates to proportional changes in the Marshallian demand. 

However, this is not the legitimate Hicksian own and cross price elasticities. Hicksian demand function or compensated demand function for a good is his quantity demanded as part of the solution to minimizing his expenditure on all goods while delivering \textbf{a fixed level of utility}. This contradicts with the underlining assumption of the logit demand function. 
\begin{equation*}
Pr\bigg(i \rightarrow j \bigg) = Pr\bigg( U_{ij} \geq U_{ik} \forall k \neq j \in \mathcal{J}_t \bigg )
\end{equation*}

As a consequence, there is no utility level $\bar U$ which could serve as the fixed threshold in the sense that $Pr\bigg(i \rightarrow j \bigg) = Pr\bigg( U_{ij} \geq \bar U \bigg )$. In other words, $U_{ij} \geq \bar U$ cannot rule out the possibility that $U_{ik} > U_{ij} \geq \bar U$. In summary, the proposed price elasticities are illegitimate Hiscksian own and cross elasticities as only the \textbf{relative} utilities matter in the logit demand models, rather than the \textbf{absolute} levels. 


\section{Linear Probability Model}
\addtolength{\jot}{3pt}

\subsection*{Question (a)}
\begin{equation}
\begin{split}
P_i & = Pr\bigg\{ y_i = x_i^\prime \beta + u_i > 0 \bigg| x_i \bigg\} \\
& = Pr\bigg\{ u_i > - x_i^\prime \beta \bigg| x_i \bigg\} \\
\end{split}
\end{equation}

Given $u_i \sim^{iid} Uniform[-0.5, 0.5]$, we have 
\begin{equation}
\begin{split}
P_i & = Pr\bigg\{ u_i > - x_i^\prime \beta \bigg| x_i \bigg\} \\
& = 0.5 + x_i^\prime \beta
\end{split}
\end{equation}

The error term in the linear probability model is
\begin{equation*}
e_i = d_i - P_i =
\begin{cases}
0.5 - x_i^\prime \beta & \text{if  } y_i = x_i^\prime \beta + u_i > 0 \\
-0.5 - x_i^\prime \beta & \text{otherwise  } 
\end{cases}
\end{equation*}

Then we derive the conditional expectation of the error terms as 
\begin{equation*}
\begin{split}
\pmb{E}\bigg(e_i \bigg| x_i\bigg) & = Pr\bigg\{ d_i = 1 \bigg\} \pmb{E}\bigg(e_i \bigg| x_i, d_i = 1\bigg)  + Pr\bigg\{ d_i = 0 \bigg\} \pmb{E}\bigg(e_i \bigg| x_i, d_i = 0\bigg)\\
& = \bigg( 0.5 - x_i^\prime \beta \bigg) \bigg( 0.5 + x_i^\prime \beta \bigg) + \bigg(-0.5 - x_i^\prime \beta \bigg) \bigg( 0.5 - x_i^\prime \beta \bigg) \\
& = 0
\end{split}
\end{equation*}

\subsection*{Question (b)}
Then we derive the conditional variance of the error terms as 
\begin{equation}
\begin{split}
\sigma_i^2 = \pmb{Var}\bigg(e_i \bigg| x_i\bigg) & = \pmb{E} \bigg(e_i^2 \bigg| x_i\bigg) + \Bigg\{ \pmb{E} \bigg(e_i \bigg| x_i\bigg) \Bigg\}^2\\
& = \pmb{E} \bigg(e_i^2 \bigg| x_i\bigg) \\
& = Pr\bigg\{ d_i = 1 \bigg\} \pmb{E}\bigg(e_i^2 \bigg| x_i, d_i = 1\bigg)  + Pr\bigg\{ d_i = 0 \bigg\} \pmb{E}\bigg(e_i^2 \bigg| x_i, d_i = 0\bigg)\\
& = \bigg( 0.5 - x_i^\prime \beta \bigg)^2 \bigg( 0.5 + x_i^\prime \beta \bigg) + \bigg(-0.5 - x_i^\prime \beta \bigg)^2 \bigg( 0.5 - x_i^\prime \beta \bigg) \\
& = 0.25 - \bigg(x_i^\prime \beta \bigg)^2
\end{split}
\end{equation}

\subsection*{Question (c)}

\begin{equation}
\begin{split}
\pmb{Cov}\bigg(e_i, e_j \bigg| x_i, x_j \bigg) & = \pmb{E} \bigg(e_i, e_j \bigg| x_i, x_j \bigg) + \Bigg\{ \pmb{E} \bigg(e_i \bigg| x_i\bigg) \pmb{E} \bigg(e_j \bigg| x_j \bigg) \Bigg\}\\
& = \pmb{E} \bigg(e_i, e_j \bigg| x_i, x_j \bigg) \\
& = Pr\bigg\{ d_i = 1, d_j = 1 \bigg\} \pmb{E}\bigg(e_i e_j \bigg| x_i, x_j, d_i = 1, d_j = 1 \bigg) \\ 
& + Pr\bigg\{ d_i = 1, d_j = 0 \bigg\} \pmb{E}\bigg(e_i e_j \bigg| x_i, x_j, d_i = 1, d_j = 0 \bigg) \\
& + Pr\bigg\{ d_i = 0, d_j = 1 \bigg\} \pmb{E}\bigg(e_i e_j \bigg| x_i, x_j, d_i = 0, d_j = 1 \bigg) \\
& + Pr\bigg\{ d_i = 1, d_j = 1 \bigg\} \pmb{E}\bigg(e_i e_j \bigg| x_i, x_j, d_i = 0, d_j = 0 \bigg) \\
& = 0
\end{split}
\end{equation}

Combining (5)(6), we know that the error terms are heterogeneous and uncorrelated. 
\begin{equation}
\pmb{\Omega} = Diag\bigg(\sigma_i^2 \bigg)
\end{equation}

\addtolength{\jot}{-3pt}
The regression specification of the linear probability model is as follows
\begin{equation*}
d_i = P_i + e_i = 0.5 + x_i^\prime \beta + e_i
\end{equation*}

The Ordinary Least-Square estimator
\begin{equation*}
\hat{\beta}_{OLS} = \bigg(\pmb{X}^\prime \pmb{X} \bigg)^{-1} \bigg(\pmb{X}^\prime (\pmb{d - 0.5} ) \bigg)
\end{equation*}

The estimated asymptotic variance matrix of $\hat \beta_{OLS}$ with heteroskedasticity robust standard errors 
\begin{equation}
\hat V [\hat \beta_{OLS}] = \bigg(\pmb{X}^\prime \pmb{X}\bigg)^{-1} \bigg(\pmb{X}^\prime \pmb{\hat \Omega} \pmb{X} \bigg)  \bigg(\pmb{X}^\prime \pmb{X}\bigg)^{-1}
\end{equation}
where $\pmb{\hat \Omega} = Diag(\hat e_i^2)$ and $\hat e_i = d_i - 0.5 - x_i^\prime \hat \beta_{OLS}$. 

Given $e_i \bigg| x_i \rightarrow^d N\bigg(0, \pmb{\Omega} \bigg)$ and $\pmb{\Omega} = Diag\bigg(\sigma_i^2 \bigg)$, we have 
\begin{equation*}
plim \hat V [\hat \beta_{OLS}] = plim V [\hat \beta_{OLS}] = \bigg(\pmb{X}^\prime \pmb{X}\bigg)^{-1} \bigg(\pmb{X}^\prime \pmb{\Omega} \pmb{X} \bigg)  \bigg(\pmb{X}^\prime \pmb{X}\bigg)^{-1}
\end{equation*}

\subsection*{Question (d)}
Given that the errors are heterogeneous and uncorrelated, we specify $V[e \big| x ] = exp(x^\prime \gamma)$. 

Then use the non-linear least-square regression of $\hat e_i^2 = (d_i - 0.5 - x_i^\prime \hat \beta)^2$ on $exp(x^\prime \gamma)$ to derive $\hat \gamma$, which is the consistent estimator of $\gamma$. The estimator error matrix is 
\begin{equation*}
\pmb{\hat \Omega} = \pmb{\hat \Omega}(\hat \gamma) = exp(x^\prime \hat \gamma)
\end{equation*} 

We have the feasible generalized least-square estimator $\hat{\beta}_{FGLS}$, which is more efficient than the given $\hat{\beta}$ as 
\begin{equation*}
\hat{\beta}_{FGLS} = \bigg(\pmb{X}^\prime \pmb{\hat \Omega}^{-1} \pmb{X} \bigg)^{-1} \bigg(\pmb{X}^\prime \pmb{\hat \Omega}^{-1} (\pmb{d - 0.5} ) \bigg)
\end{equation*}

The estimated asymptotic variance of $\hat{\beta}_{FGLS}$ is 
\begin{equation}
\hat V [\hat \beta_{FGLS}] = \bigg(\pmb{X}^\prime \pmb{\hat \Omega}^{-1} \pmb{X} \bigg)^{-1}  
\end{equation}

\subsection*{Question (e)}
Given that $\pmb{\Omega}(\gamma)$ is correctly specified and that $\hat gamma$ is consistent for $\gamma$, we have (Cameron \& Trivedi, page 82)
\begin{equation*}
\sqrt{N} \bigg( \hat \beta_{FGLS} - \beta \bigg) \rightarrow^{d} N\bigg[0, \bigg( plim N^{-1} \pmb{X}^\prime \pmb{\Omega}^{-1} \pmb{X} \bigg)^{-1} \bigg]
\end{equation*}

This implies that 
\begin{itemize}
	\item $\hat \beta_{FGLS}$ is a consistent estimator of $\beta$. $\hat \beta_{FGLS} \rightarrow_p \beta$ as $n \rightarrow \infty$
	\item $\hat \beta_{FGLS}$ is second-moment efficient as $\hat \beta_{FGLS}$ has the same limiting variance matrix as $\hat \beta_{GLS}$. 
\end{itemize}

In summary, $\hat \beta_{FGLS}$ possesses the same asymptotic properties as $\hat \beta_{MLE}$. 

\section{Multinomial Logit and MLE Estimation}
\subsection*{Question (a)}
The likelihood function of the binary choice models for individual $i$ is 
\begin{equation*}
L\bigg(\{ y_{i,j}, x_j \}_{j \in J} \bigg| \beta \bigg) = \prod_{i = 1}^{N} \prod_{j \in J} \bigg[ \frac{exp(x_j^\prime \beta)}{ \sum_{k = 0}^{J} exp(x_k^\prime \beta)} \bigg]^{y_{i,j}} 
\end{equation*}


The log-likelihood function of observing $\{ y_{i,j}, x_j \}_{j \in J}$ for individuals $i = 1, 2, ..., N$ is 
\begin{equation*}
lnL\bigg(\{ y_{i,j}, x_j \}_{j \in J} \bigg| \beta \bigg) = \sum_{i = 1}^{N} \sum_{j \in J} \bigg\{  y_{i,j} ln \big( \frac{exp(x_j^\prime \beta)}{ \sum_{k \in J} exp(x_k^\prime \beta)} \big) \bigg\}
\end{equation*}

\subsection*{Question (b)}
Denote $G(x_j^\prime \beta) = \frac{exp(x_j^\prime \beta)}{ \sum_{k \in J} exp(x_k^\prime \beta)}$ and 
\begin{equation*}
\begin{split}
g(x_j^\prime \beta) & = \frac{ d G(x_j^\prime \beta) }{d \beta } \\
& = \frac{ exp(x_j^\prime \beta) \sum_{k \in J} \bigg[ (x_j - x_k) exp(x_k^\prime \beta) \bigg]  }{ \bigg(\sum_{k \in J} exp(x_k^\prime \beta) \bigg)^2  } \\
\end{split}
\end{equation*}

Then we can rewrite $ln L\bigg(\{ y_{i,j}, x_j \}_{j \in J} \bigg| \beta \bigg) $ as 
\begin{equation*}
lnL\bigg(\{ y_{i,j}, x_j \}_{j \in J} \bigg| \beta \bigg) = \sum_{i = 1}^{N} \sum_{j \in J} \bigg\{  y_{i,j} ln \bigg( G(x_j^\prime \beta) \bigg) \bigg\}
\end{equation*}

Taking derivatives of $lnL\bigg(\{ y_{i,j}, x_j \}_{j = 0}^J \bigg| \beta \bigg)$ with respects to $\beta$, we can get the score function of the likelihood of observing the data $\{ y_{i,j}, x_j \}_{j \in J}$ for individuals $i = 1, 2, ..., N$ as follows. 
\begin{equation*}
\begin{split}
s\bigg(\beta \bigg| \{ y_{i,j}, x_j \}_{j \in J} \bigg) & = \frac{d lnL \bigg(\{ y_{i,j}, x_j \}_{j \in J} \bigg| \beta \bigg) }{d \beta} \\
& = \sum_{i = 1}^{N}\sum_{j \in J} \bigg\{  y_{i,j} \frac{ g(x_j^\prime \beta) }{ G(x_j^\prime \beta) }  \bigg\}
\end{split}
\end{equation*}

\subsection*{Question (c)}
The information matrix of the (unconditional) likelihood of observing the data for
individuals $i = 1, 2, ..., N$ 
\begin{equation*}
I(\beta) = E_{\beta}\Bigg( s\bigg(\beta \bigg| \{ y_{i,j}, x_j \}_{j \in J} \bigg) s\bigg(\beta \bigg| \{ y_{i,j}, x_j \}_{j \in J} \bigg)^\prime \Bigg) = \sum_{i = 1}^{N}\sum_{j \in J} E_{\beta}\bigg\{  y_{i,j} \frac{ g(x_j^\prime \beta) g(x_j^\prime \beta)^\prime}{ G^2(x_j^\prime \beta) }  \bigg\}
\end{equation*}

Under mild regularity conditions, 
\begin{equation*}
I(\beta) = - E_{\beta} \Bigg( \frac{\partial}{\partial \beta} s\bigg(\beta \bigg| \{ y_{i,j}, x_j \}_{j \in J} \bigg) \Bigg) = - E_{\beta} \Bigg( \frac{\partial^2}{\partial \beta^2} lnL\bigg(\{ y_{i,j}, x_j \}_{j \in J} \bigg| \beta \bigg) \Bigg)
\end{equation*}


\end{document}
